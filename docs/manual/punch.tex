\subsection[Card Punch]{CARD PUNCH}

This section describes the configuration and operation of the simulated Card Punch.

The card punch operates by writing a simulated card deck to a spool file. Each deck
will be placed in it's own spool file. The location of these spool files is controlled
by the "path" global option (see below).

The file will contain all banner, lace and trailer cards that Multics prints. The lace
cards are parsed to extract job information that is used to name the spool file.

Note that there is a utility "punutil" that can be used to extract the various cards from 
the deck in a more usable format (such as ASCII).

\subsubsection[Card Punch File Format]{CARD PUNCH FILE FORMAT}

Each column in a punched card has twelve rows, designated from top to bottom:

\begin{adjustwidth}{5ex}{1ex}
    \texttt{\& - 0 1 2 3 4 5 6 7 8 9}
\end{adjustwidth}

Given the above, an 80 column punched card would then look like this:

\begin{adjustwidth}{5ex}{1ex}
	\begin{tabular}{|c|c|c|c|c|c|c|c|c|c|c|}
        \cline{1-5} \cline{7-11}
		\texttt{\&} & \texttt{\&} & \texttt{\&} & \texttt{\&} & \texttt{\&} & & \texttt{\&} & \texttt{\&} & \texttt{\&} & \texttt{\&} & \texttt{\&} \\
        \cline{1-5} \cline{7-11}
		\texttt{-} & \texttt{-} & \texttt{-} & \texttt{-} & \texttt{-} & & \texttt{-} & \texttt{-} & \texttt{-} & \texttt{-} & \texttt{-} \\
        \cline{1-5} \cline{7-11}
		\texttt{0} & \texttt{0} & \texttt{0} & \texttt{0} & \texttt{0} & & \texttt{0} & \texttt{0} & \texttt{0} & \texttt{0} & \texttt{0} \\
        \cline{1-5} \cline{7-11}
		\texttt{1} & \texttt{1} & \texttt{1} & \texttt{1} & \texttt{1} & & \texttt{1} & \texttt{1} & \texttt{1} & \texttt{1} & \texttt{1} \\
        \cline{1-5} \cline{7-11}
		\texttt{2} & \texttt{2} & \texttt{2} & \texttt{2} & \texttt{2} & & \texttt{2} & \texttt{2} & \texttt{2} & \texttt{2} & \texttt{2} \\
        \cline{1-5} \cline{7-11}
		\texttt{3} & \texttt{3} & \texttt{3} & \texttt{3} & \texttt{3} & \texttt{...} & \texttt{3} & \texttt{3} & \texttt{3} & \texttt{3} & \texttt{3} \\
        \cline{1-5} \cline{7-11}
		\texttt{4} & \texttt{4} & \texttt{4} & \texttt{4} & \texttt{4} & & \texttt{4} & \texttt{4} & \texttt{4} & \texttt{4} & \texttt{4} \\
        \cline{1-5} \cline{7-11}
		\texttt{5} & \texttt{5} & \texttt{5} & \texttt{5} & \texttt{5} & & \texttt{5} & \texttt{5} & \texttt{5} & \texttt{5} & \texttt{5} \\
        \cline{1-5} \cline{7-11}
		\texttt{6} & \texttt{6} & \texttt{6} & \texttt{6} & \texttt{6} & & \texttt{6} & \texttt{6} & \texttt{6} & \texttt{6} & \texttt{6} \\
        \cline{1-5} \cline{7-11}
		\texttt{7} & \texttt{7} & \texttt{7} & \texttt{7} & \texttt{7} & & \texttt{7} & \texttt{7} & \texttt{7} & \texttt{7} & \texttt{7} \\
        \cline{1-5} \cline{7-11}
		\texttt{8} & \texttt{8} & \texttt{8} & \texttt{8} & \texttt{8} & & \texttt{8} & \texttt{8} & \texttt{8} & \texttt{8} & \texttt{8} \\
        \cline{1-5} \cline{7-11}
		\texttt{9} & \texttt{9} & \texttt{9} & \texttt{9} & \texttt{9} & & \texttt{9} & \texttt{9} & \texttt{9} & \texttt{9} & \texttt{9} \\
        \cline{1-5} \cline{7-11}
	\end{tabular}
\end{adjustwidth}

When representing a card column, the most obvious way to do it is to assign a single bit per punch. This 
means that there are 12 bits for each column. When looking at a hexadecimal dump of a spool file, each column
occupies three 4 bit nibbles. The following representation is big-endian:

\begin{adjustwidth}{5ex}{1ex}
	\begin{tabular}{|c|c|c|c|c|c|c|c|c|c|c|c|c|c|c|c|c|c|c|c|c|c|c|c|}
        \multicolumn{12}{|c|}{Column 1} & \multicolumn{12}{|c|}{Column 2} \\
        \hline
        \texttt{\&} & \texttt{-} & \texttt{0} & \texttt{1} & \texttt{2} & \texttt{3} & \texttt{4} & \texttt{5} & \texttt{6} & \texttt{7} & \texttt{8} & \texttt{9} & \texttt{\&} & \texttt{-} & \texttt{0} & \texttt{1} & \texttt{2} & \texttt{3} & \texttt{4} & \texttt{5} & \texttt{6} & \texttt{7} & \texttt{8} & \texttt{9} \\
        \hline
        \multicolumn{8}{|c|}{Byte 1} & \multicolumn{8}{|c|}{Byte 2} & \multicolumn{8}{|c|}{Byte 3} \\
        \texttt{7} & \texttt{6} & \texttt{5} & \texttt{4} & \texttt{3} & \texttt{2} & \texttt{1} & \texttt{0} & \texttt{7} & \texttt{6} & \texttt{5} & \texttt{4} & \texttt{3} & \texttt{2} & \texttt{1} & \texttt{0} & \texttt{7} & \texttt{6} & \texttt{5} & \texttt{4} & \texttt{3} & \texttt{2} & \texttt{1} & \texttt{0} \\
	\end{tabular}
\end{adjustwidth}

\newpage

\subsubsection[Card Punch Parameters]{CARD PUNCH PARAMETERS}

This section describes the parameters that can be set for the Card Punches.

The Global Parameters apply to all card punch units while the Unit Parameters apply to a single card punch unit.

\textbf{Global Parameters}

Global Parameters are set using the "\texttt{SET PUN XXXX=yyyy}" simulator command (\texttt{XXXX} is the parameter being set
and \texttt{yyyy} is the value to set).

The "\texttt{SHOW PUN XXXX}" command will show the current value of a global parameter.

\textbf{PATH}

\begin{adjustwidth}{5ex}{1ex}
    Sets the path that will be used as the location to write punch spool files. When not set, all punch spool files will be written
    to the simulator's current directory. When this path is set, the simulator expects there to be a subdirectory for each named punch 
    unit (the simulator will \textbf{not} create any directories).  For example, if the PATH is set to "/home/tom/punches" and three
    card punch units are defined (puna, punb and punc), then the following directory structure must exist for punch jobs to be output properly:

    \begin{adjustwidth}{5ex}{1ex}
        \texttt{/home/tom/punches} \\
        \texttt{/home/tom/punches/puna} \\
        \texttt{/home/tom/punches/punb} \\
        \texttt{/home/tom/punches/punc} \\
    \end{adjustwidth}
\end{adjustwidth}

\subsubsection[Card Punch Options]{CARD PUNCH OPTIONS}

Options are set on a per device basis with the "\texttt{SET PUNn CONFIG=XXXX=yyyy}" simulator command (\texttt{n} is the unit number with 0 = A, 1 = B, etc,
\texttt{XXXX} is the name of the option and \texttt{yyyy} is the option value).

\textbf{LOGCARDS}

\begin{adjustwidth}{5ex}{1ex}
    When this option is turned on, a significant amount of card punch diagnostic output will be displayed on the console. This includes
    an "ASCII Art" form of the punched cards where asterisks (*) are used to represent punched holes. It is recommended to only turn on
    this option if you are attempting to diagnose a card punch issue.

    Valid option values are:

    \begin{adjustwidth}{5ex}{1ex}
        \begin{tabular}{cc}
            \hline
            \texttt{0}  & \\
            \texttt{off} & Turn off logging of card punch diagnostic data \\ 
            \texttt{disable} & \\
            \hline
            \texttt{1} & \\
            \texttt{on} & Turn on logging of card punch diagnostic data \\ 
            \texttt{enable} & \\
            \hline
        \end{tabular}
    \end{adjustwidth}
\end{adjustwidth}

