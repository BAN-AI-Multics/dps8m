% SPDX-License-Identifier: LicenseRef-DPS8M-Doc AND LicenseRef-CF-GAL
% SPDX-FileCopyrightText: 2019-2021 Dean S. Anderson
% SPDX-FileCopyrightText: 2021-2022 The DPS8M Development Team

\subsection[Tape Drives]{TAPE DRIVES}

This section describes the configuration and operation of the simulated Tape Drives (TAP).

\subsubsection[Operator Console Options]{TAPE DRIVE PARAMETERS}

This section describes the parameters that can be set for the Tape Drives.

The Global Parameters apply to all tape drives while the Unit Parameters apply to a single drive unit.

\textbf{Global Parameters}

Global Parameters are set using the "\texttt{SET TAPE XXXX=yyyy}" simulator command (\texttt{XXXX} is the parameter being
set and \texttt{yyyy} is the value to set).

The "\texttt{SHOW TAPE XXXX}" command will show the current value of a global parameter.

\textbf{ADD\_PATH}

\begin{adjustwidth}{5ex}{1ex}
	Adds a new tape search path at the end of the current search path list. The parameter format is:
\begin{adjustwidth}{5ex}{1ex}
	\texttt{prefix=directory}
\end{adjustwidth}
	For example, to look for tapes with a volume name starting with "BK" in a directory "./tapes/backups":
\begin{adjustwidth}{5ex}{1ex}
	\texttt{SET TAPE ADD\_PATH=BK=./tapes/backups}
\end{adjustwidth}
	Note that when using relative paths, the starting point is the simulator's current directory.
	Also note that if a \texttt{SET DEFAULT\_PATH} is done, all previous \textbf{ADD\_PATH} entries
	will be removed.
	
	Adding paths can't be done until the \textbf{DEFAULT\_PATH} has first been set.
\end{adjustwidth}

\textbf{CAPACITY\_ALL}

\begin{adjustwidth}{5ex}{1ex}
	Sets the maximum size that can be written to a tape file for all tape drives.
\end{adjustwidth}

\textbf{DEFAULT\_PATH}

\begin{adjustwidth}{5ex}{1ex}
	Sets the default path to use when searching for tape files. Note that setting this option will clear
	out any previous \textbf{ADD\_PATH} parameters.
	
	If \textbf{DEFAULT\_PATH} is not set, the simulator will default to look for tape files in its current directory.

	For example, to look for tapes in a directory "./tapes":
\begin{adjustwidth}{5ex}{1ex}
	\texttt{SET TAPE DEFAULT\_PATH=./tapes}
\end{adjustwidth}
	Note that when using relative paths, the starting point is the simulator's current directory.
\end{adjustwidth}

\textbf{NUNITS}

\begin{adjustwidth}{5ex}{1ex}
	Sets the maximum number of tape drives available.
\end{adjustwidth}


\textbf{Unit Parameters}

Unit parameters are set on a per Tape Drive basis. Unit parameters are set using
the "\texttt{SET TAPEn XXXX=yyyy}" simulator command (\texttt{n} is the tape drive number, \texttt{XXXX} is
the parameter being set and \texttt{yyyy} is the value to set).

\textbf{NAME}

\begin{adjustwidth}{5ex}{1ex}
	Set the name of a specific tape drive. For example, to set the name of tape drive 1 to "tapa\_01":
\begin{adjustwidth}{5ex}{1ex}
	\texttt{SET TAPE1 NAME=tapa\_01}
\end{adjustwidth}
\end{adjustwidth}


\subsubsection[Tape Drive Options]{TAPE DRIVE OPTIONS}

The tape drives have no configuration options.

\subsubsection[Creating a Search Table]{CREATING A SEARCH TABLE}

When the search table is evaluated during a tape file lookup, it is processed in order with a first
match being accepted. For example, if the following commands are given:

\begin{adjustwidth}{5ex}{1ex}
	\texttt{SET TAPE DEFAULT\_PATH=./tapes/general} \\
	\texttt{SET TAPE ADD\_PATH=B=./tapes/billing}
	\texttt{SET TAPE ADD\_PATH=BK=./tapes/backups} \\
\end{adjustwidth}

a tape with a volume name of "BK001" will end up in the ./tapes/billing directory and not the
./tapes/backups directory. Since the name starts with a "B", the first entry in the search table
will be selected.

To get the intended effect, the commands should be reorder like this:

\begin{adjustwidth}{5ex}{1ex}
	\texttt{SET TAPE DEFAULT\_PATH=./tapes/general} \\
	\texttt{SET TAPE ADD\_PATH=BK=./tapes/backups} \\
	\texttt{SET TAPE ADD\_PATH=B=./tapes/billing}
\end{adjustwidth}

this way the names starting with "BK" will be checked first followed by the more general "B".

\textbf{Important Note:} The simulator "attach" command (most often used in .ini files to mount
the boot tape) is not aware of the alternate search directories so, if a directory is needed for
an attach command, the path must be specified on the command:

\begin{adjustwidth}{5ex}{1ex}
	\texttt{attach -r tape0 ./tapes/12.7/12.7MULTICS.tap}
\end{adjustwidth}
