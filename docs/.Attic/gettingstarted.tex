\section[Getting Started]{GETTING STARTED}

This section covers getting started with the simulator. First, getting a copy of the simulator (build or download) will be discussed.
Then a review of a quick Multics startup will be shown.

\subsection[Building the Simulator from Git]{BUILDING THE SIMULATOR FROM GIT}

Building the simulator has been tested on a number of platforms. 

\textbf{\textcolor{red}{DRAFT NOTE: All of the instructions below are for Release 2.0 of the simulator and need to be verified and/or updated for 2.1.}}

\subsubsection[Building dps8m under Ubuntu]{BUILDING DPS8M UNDER UBUNTU}

\textbf{\textcolor{red}{This needs to be tested}}

\textbf{Install needed packages:}

\begin{adjustwidth}{5ex}{1ex}
	\texttt{(for 16.04):    sudo apt install git clang} \\
    \texttt{(for 18.04):    sudo apt install git clang libtool m4 automake}
\end{adjustwidth}  

\textbf{Build 'libuv'}

\begin{adjustwidth}{5ex}{1ex}
    \texttt{git clone https://github.com/libuv/libuv.git} \\
    \texttt{cd libuv} \\
    \texttt{git checkout v1.23.0} \\
    \texttt{sh autogen.sh} \\
    \texttt{./configure} \\
    \texttt{make} \\
    \texttt{sudo make install} \\
    \texttt{cd ..} \\
\end{adjustwidth}  

\textbf{Build dps8}

\begin{adjustwidth}{5ex}{1ex}
    \texttt{git clone https://gitlab.com/dps8m/dps8m} \\
    \texttt{cd dps8m} \\
    \texttt{git checkout R2.0} \\
    \texttt{make LOCKLESS=1} \\
\end{adjustwidth}  

\textbf{Running the Simulator}

\begin{adjustwidth}{5ex}{1ex}
    \texttt{(Copy 'src/dps8/dps8' to the desired working directory)} \\
    \texttt{./dps8 [boot script]} \\
\end{adjustwidth}  

\newpage

\subsubsection[Building dps8m under MacOS]{BUILDING DPS8M UNDER MACOS}

\textbf{\textcolor{red}{This needs to be update/tested}}

\textbf{Install needed packages:}

\begin{adjustwidth}{5ex}{1ex}
	\texttt{brew install git clang libtool m4 automake} \\
\end{adjustwidth}  

\textbf{Build 'libuv'}

\begin{adjustwidth}{5ex}{1ex}
    \texttt{git clone https://github.com/libuv/libuv.git} \\
    \texttt{cd libuv} \\
    \texttt{git checkout v1.23.0} \\
    \texttt{sh autogen.sh} \\
    \texttt{./configure} \\
    \texttt{make LOCKLESS=1} \\
    \texttt{sudo make install} \\
    \texttt{cd ..} \\
\end{adjustwidth}  

\textbf{Build dps8}

\begin{adjustwidth}{5ex}{1ex}
    \texttt{git clone https://gitlab.com/dps8m/dps8m} \\
    \texttt{cd dps8m} \\
    \texttt{git checkout R2.0} \\
    \texttt{make LOCKLESS=1} \\
\end{adjustwidth}  

\textbf{Running the Simulator}

\begin{adjustwidth}{5ex}{1ex}
    \texttt{(Copy 'src/dps8/dps8' to the desired working directory)} \\
    \texttt{./dps8 [boot script]} \\
\end{adjustwidth}  

\newpage

\subsubsection[Building dps8m under CentOS]{BUILDING DPS8M UNDER CENTOS}

\textbf{\textcolor{red}{This needs to be tested}}

\textbf{Install needed packages:}

\begin{adjustwidth}{5ex}{1ex}
	\texttt{sudo yum install git automake libtool clang} \\
\end{adjustwidth}  

\textbf{Build 'libuv'}

\begin{adjustwidth}{5ex}{1ex}
    \texttt{git clone https://github.com/libuv/libuv.git} \\
    \texttt{cd libuv} \\
    \texttt{git checkout v1.23.0} \\
    \texttt{sh autogen.sh} \\
    \texttt{./configure} \\
    \texttt{make} \\
    \texttt{sudo make install} \\
    \texttt{cd ..} \\
\end{adjustwidth}  

NOTE: For some reason libuv is not being found in its default installation location of /usr/local/lib and it was necessary do the following for it to be found:

\begin{adjustwidth}{5ex}{1ex}
    \texttt{sudo ln -s /usr/local/lib/libuv.so.1.0.0 /usr/lib64/libuv.so.1} \\
\end{adjustwidth}  

\textbf{Build dps8}

\begin{adjustwidth}{5ex}{1ex}
    \texttt{git clone https://gitlab.com/dps8m/dps8m} \\
    \texttt{cd dps8m} \\
    \texttt{git checkout R2.0} \\
    \texttt{make LOCKLESS=1} \\
\end{adjustwidth}  

\textbf{Running the Simulator}

\begin{adjustwidth}{5ex}{1ex}
    \texttt{(Copy 'src/dps8/dps8' to the desired working directory)} \\
    \texttt{./dps8 [boot script]} \\
\end{adjustwidth}  

\newpage

\subsubsection[Building dps8m under Fedora 30 Linux For Windows]{BUILDING DPS8M UNDER FEDORA 30 LINUX FOR WINDOWS}

\textbf{\textcolor{red}{This needs to be tested}}

\textbf{Install needed packages:}

\begin{adjustwidth}{5ex}{1ex}
	\texttt{sudo dnf install git mingw64-gcc mingw64-libgnurx libtool make} \\
    \texttt{sudo dnf update perl-Errno} \\
\end{adjustwidth}  

\textbf{Build 'libuv'}

\begin{adjustwidth}{5ex}{1ex}
    \texttt{git clone https://github.com/libuv/libuv.git} \\
    \texttt{cd libuv} \\
    \texttt{git checkout v1.23.0} \\
    \texttt{sh autogen.sh} \\
    \texttt{mingw64-configure} \\
    \texttt{make} \\
    \texttt{sudo make install} \\
    \texttt{cd ..} \\
\end{adjustwidth}  

\textbf{Build dps8}

\begin{adjustwidth}{5ex}{1ex}
    \texttt{git clone https://gitlab.com/dps8m/dps8m} \\
    \texttt{cd dps8m} \\
    \texttt{git checkout R2.0} \\
    \texttt{make CROSS=MINGW64 LOCKLESS=1} \\
\end{adjustwidth}  

\textbf{Running the Simulator}

\begin{adjustwidth}{5ex}{1ex}
    \texttt{Copy these files to your windows machine:} \\
    \texttt{dps8.exe} \\
    \texttt{/usr/x86\_64-w64-mingw32/sys-root/mingw/bin/libwinpthread-1.dll} \\
    \texttt{/usr/x86\_64-w64-mingw32/sys-root/mingw/bin/libuv-1.dll} \\
\end{adjustwidth}  

\newpage

\subsubsection[Building dps8m under Windows with MSYS2]{BUILDING DPS8M UNDER WINDOWS WITH MSYS2}

\textbf{\textcolor{red}{This needs to be tested}}

\textbf{Install MSYS2 and packages:}

In your browser, go to https://msys2.github.io/ \\

Download and run the msys2\ x86\_64\ installer. Follow the installation
instructions on the webpage. \\

After step 7 ("Now Pacman is fully committed…"), fetch the dps8m
code, needed libraries and packages by entering: \\

\begin{adjustwidth}{5ex}{1ex}
	\texttt{git clone git://git.code.sf.net/p/dps8m/code dps8m-code} \\
    \texttt{git clone https://github.com/libuv/libuv.git} \\
    \\
    \texttt{pacman -S mingw-w64-x86\_64-gcc} \\
    \texttt{pacman -S mingw-w64-x86\_64-binutils} \\
    \texttt{pacman -S mingw-w64-x86\_64-libtool} \\
    \texttt{pacman -S mingw-w64-x86\_64-make} \\
    \texttt{pacman -S mingw-w64-x86\_64-dlfcn} \\
    \texttt{pacman -S mingw-w64-x86\_64-diffutils} \\
    \texttt{pacman -S mingw-w64-x86\_64-gettext} \\
    \texttt{pacman -S mingw-w64-x86\_64-libgnurx} \\
    \texttt{pacman -S mingw-w64-x86\_64-binutils} \\
    \texttt{pacman -S automake} \\
    \texttt{pacman -S autoconf} \\
    \texttt{pacman -S unzip} \\
    \texttt{pacman -S zip} \\
\end{adjustwidth}  

\textbf{Start a MSYS2 MinGW window and build the code:}

\begin{adjustwidth}{5ex}{1ex}
    \texttt{Start -> All programs -> MSYS2 64 bit -> MSYS2 MINGW 64 bit} \\
\\
    \texttt{cd libuv} \\
    \texttt{git checkout v1.23.0} \\
    \texttt{sh autogen.sh} \\
    \texttt{./configure} \\
    \texttt{mingw32-make.exe MAKE=mingw32-make.exe} \\
\\
    \texttt{cd} \\
    \texttt{cd dps8-code} \\
    \texttt{git checkout R2.0} \\
    \texttt{git fetch} \\
    \texttt{mingw32-make.exe} \\
\end{adjustwidth}  

\textbf{Running the Simulator}

Start a Windows command shell:

\begin{adjustwidth}{5ex}{1ex}
    \texttt{Start -> All programs -> Accessories -> Command Prompt} \\
\end{adjustwidth}  

Change directory to where the simulator is to be run from and copy the needed files:

\begin{adjustwidth}{5ex}{1ex}
    \texttt{cd <wherever>} \\
    \texttt{copy c:\textbackslash{}msys64\textbackslash{}home\textbackslash{}Admin\textbackslash{}dps8m-code\textbackslash{}rc\textbackslash{}dps8\textbackslash{}dps8.exe} \\
    \texttt{copy c:\textbackslash{}msys64\textbackslash{}mingw64\textbackslash{}bin\textbackslash{}libwinpthread-1.dll} \\
    \texttt{copy c:\textbackslash{}msys64\textbackslash{}mingw64\textbackslash{}bin\textbackslash{}libuv-1.dll} \\
\end{adjustwidth}  

Run Multics:

\begin{adjustwidth}{5ex}{1ex}
    \texttt{.\textbackslash{}dps8.exe <boot.ini>} \\
\end{adjustwidth}  

\newpage

\subsubsection[Building dps8m under Raspbian]{BUILDING DPS8M UNDER RASPIAN}

\textbf{\textcolor{red}{This needs to be tested}}

\textbf{Install needed packages:}

\begin{adjustwidth}{5ex}{1ex}
	\texttt{sudo apt-get update} \\
    \texttt{sudo apt-get install git libtool automake clang}
\end{adjustwidth}  

\textbf{Build 'libuv'}

\begin{adjustwidth}{5ex}{1ex}
    \texttt{git clone https://github.com/libuv/libuv.git} \\
    \texttt{cd libuv} \\
    \texttt{git checkout v1.23.0} \\
    \texttt{sh autogen.sh} \\
    \texttt{./configure --prefix=/usr} \\
    \texttt{make} \\
    \texttt{sudo make install} \\
    \texttt{cd ..} \\
\end{adjustwidth}  

\textbf{Build dps8}

\begin{adjustwidth}{5ex}{1ex}
    \texttt{git clone https://gitlab.com/dps8m/dps8m} \\
    \texttt{cd dps8m} \\
    \texttt{git checkout R2.0} \\
    \texttt{make M32=1} \\
\end{adjustwidth}  

\textbf{Running the Simulator}

\begin{adjustwidth}{5ex}{1ex}
    \texttt{(Copy 'src/dps8/dps8' to the desired working directory)} \\
    \texttt{./dps8 [boot script]} \\
\end{adjustwidth}  

