\documentclass[notitlepage]{report}
\usepackage {listings}
\usepackage{ulem}
\renewcommand{\thesection}{\arabic{part}.\arabic{section}}
\makeatletter
\@addtoreset{section}{part}
\makeatother 

\begin{document}
\title{Emulation Implementation Notes}
%\author{Charles Anthony and others}
%\date{2015-08-21}
\maketitle

% \part{Source Code Layout}
%   \section{dps8\_sys}
%     \subsection{dps8\_cable}
%   \section{dps8\_cpu}
%     \subsection{dsp8\_iefp}
%     \subsection{dps8\_addrmods}
%     \subsection{dps8\_append}
%     \subsection{dps8\_bar}
%     \subsection{dps8\_ins}
%     \subsection{dps8\_eis}
%      \subsection{dps8\_math}
%     \subsection{dps8\_opcodetable}
%     \subsection{dps8\_faults}
%     \subsection{dps8\_decimal}
%   \section{dps8\_scu}
%   \section{dps8\_iom}
%     \subsection{dps8\_console}
%     \subsection{dps8\_disk}
%     \subsection{dps8\_mt}
%     \subsection{dps8\_fnp, fnp\_ipc}
%     \subsection{dps8\_lp}
%     \subsection{dps8\_crdrdr}
%   \section{dps8\_utils}
%     \subsection{shm}
%   \section{Other}

% \part{CPU operation}
%   \section{simh}
%   \section{CPU emulation organization}
%   \section{FETCH\_cycle}
%   \section{EXEC\_cycle}
%   \section{Instruction execution: `executeInstruction'}
%   \section{Instruction execution: `doInstruction'}
%     \subsection{LCA Load Complement A}
%     \subsection{LREG Load Registers}
%     \subsection{STA Store A}
%     \subsection{STXn Store Index Register n}
%     \subsection {TRA Transfer}
%   \section{RPT/RPD}
%   \section{XEC/XED}
%   \section{EIS}
%   \section{Computed Address Formation}
%   \section{Append Unit}
%   \section{Interrupts}
%   \section{Faults}
%     \subsection{doFault()}
%     \subsection{RCU}

% \part{"As-built"}
%   \section{Variations between the hardware and the emulator}

\part{Source Code Layout}

\section{dps8\_sys} 

This module handles the abstract "entire system." The bulk of the code is 
initialization and simh command processing hooks.

\subsection{dps8\_cable}

Hardware device cabling emulation

\section{dps8\_cpu} 

The CPU emulator

\subsection{dsp8\_iefp}

CPU Memory access

\subsection{dps8\_addrmods}

CPU Address Modifcation

\subsection{dps8\_append}

CPU Append Unit

\subsection{dps8\_bar}

CPU BAR address computation

\subsection{dps8\_ins}

CPU instructions

\subsection{dps8\_eis}

CPU EIS instructions

\subsection{dps8\_math}

CPU math support routines

\subsection{dps8\_opcodetable}

CPU opcode table; drives address preparation and mode checking

\subsection{dps8\_faults}

CPU fault handling code

\subsection{dps8\_decimal}

CPU Decimal math support routines

\section{dps8\_scu}

The SCU emulator

\section{dps8\_iom}

The IOM emulator

\subsection{dps8\_console}

Operator console device

\subsection{dps8\_disk}

Disk device

\subsection{dps8\_mt}

Tape device

\subsection{dps8\_fnp, fnp\_ipc}

FNP device

\subsection{dps8\_lp}

Printer device

\subsection{dps8\_crdrdr}

Card reader device

\section{dps8\_utils}

Common utility routines

\subsection{shm}

Host shared memory management

\section{Other}

dps8\_utils Common support code \\
dps8\_clk  Unused simh clock hooks \\
dps8\_loader Segment loader for unit tests \\
dps8\_stddev Vestigial \\
dps8\_fxe Faux Multics Execution module

\part{CPU operation}

\section{simh}

When simh is running the CPU it calls sim\_instr(), which is the CPU emulator entry point.

sim\_instr() loops, executing emulated instructions until some emulation halting condition is met, or some simh component signals for a pause due to an external
event.

\section{CPU emulation organization}

sim\_instr first establishes a setjmp context; this is used primarily by RCU 
mechanism to restart instruction processing after restoring a saved system
state due to a fault or an interrupt.

The cpu emulation is written as a state machine. The longjmp parameter is 
used to setup the desired state; at initial entry, setjmp returns a zero, 
and the appropriate setup is done.

\begin{verbatim}
#define JMP\_ENTRY       0

    int val = setjmp(jmpMain);

    switch (val)
      {
        case JMP\_ENTRY:
        case JMP\_REENTRY:
            reason = 0;
            break;
        case JMP\_NEXT:
            goto nextInstruction;
        case JMP\_STOP:
            reason = STOP\_HALT;
            goto leave;
        case JMP\_SYNC\_FAULT\_RETURN:
            goto syncFaultReturn;
        case JMP\_REFETCH:
            cpu . wasXfer = false;
            setCpuCycle (FETCH\_cycle);
            break;
        case JMP\_RESTART:
            setCpuCycle (EXEC\_cycle);
            break;
        default:
          sim\_printf ("longjmp value of %d unhandled\n", val);
          goto leave;
      }

\end{verbatim}

The cpu emulation then enters a "do {...} while (reason == 0)" loop,
which cycles the CPU through its states.

The top of the loop checks the simh components for events that need
to be handled (simh\_hooks()), polls various subsystems for service requests
(emulator console commands to be processed, incoming FNP messages and 
operator console input).

\begin{verbatim}
        reason = 0;

        // Process deferred events and breakpoints
        reason = simh\_hooks ();
        if (reason)
          {
            //sim\_printf ("reason: %d\n", reason);
            break;
          }

        static uint queueSubsample = 0;
        if (queueSubsample ++ > 10240) // ~ 100Hz
          {
            queueSubsample = 0;
            scpProcessEvent ();
            fnpProcessEvent ();
            consoleProcess ();
          }
        if (check\_attn\_key ())
          console\_attn (NULL);
\end{verbatim}

It then checks for Timer Register runout, setting the group 7 fault flag
if needed. (Group 7 faults are distinguished as not resulting from
instruction execution, but external events; and has such are handled
synchronously between instruction execution steps, rather then 
interrupting mid-instruction. 

\begin{verbatim}
    bool overrun;
    UNUSED word27 rTR = getTR (& overrun);
    if (overrun)
      {
        ackTR ();
        if (switches . tro\_enable)
          setG7fault (FAULT\_TRO, 0);
      }
\end{verbatim}

Next, it then checks for lockup (the operating system has not enabled 
interrupts for more then 32 ms.), and faults if needed.

\begin{verbatim}
        lufCounter ++;
        // Assume CPU clock ~ 1Mhz. lockup time is 32 ms
        if (lufCounter > 32000)
          {
            lufCounter = 0;
            doFault (FAULT\_LUF, 0, "instruction cycle lockup");
          }
\end{verbatim}

Lastly, it checks the CPU state and branches to the appropriate code.

\begin{verbatim}
    switch (cpu . cycle)
      {
        case INTERRUPT\_cycle:
          ....
      }
\end{verbatim}

The states are:

\begin{description}
\item [FETCH\_cycle] Fetch the next instruction
\item [EXEC\_cycle] Execute an instruction
\item [INTERRUPT\_cycle] Fetch an Interrupt instruction pair
\item [INTERRUPT\_EXEC\_cycle] Execute the even instruction of an interrupt pair
\item [INTERRUPT\_EXEC2\_cycle] Execute the odd instruction of an interrupt pair
\item [FAULT\_cycle]Fetch an Fault instruction pair
\item [FAULT\_EXEC\_cycle] Execute the even instruction of a fault pair
\item [FAULT\_EXEC2\_cycle] Execute the odd instruction of a fault pair
\end{description}

The normal instruction flow is alternating FETCH and EXEC cycles.

\section{FETCH\_cycle}

The fetch cycle first checks for pending interrupts and group 7 faults, 
according to complex eligibility rules (AL39, pg 327, "Interrupt Sampling.")

\begin{verbatim}
    if ((! cpu . wasInhibited) &&
        (PPR . IC % 2) == 0 &&
        (! cpu . wasXfer) &&
        (! (cu . xde | cu . xdo | cu . rpt | cu . rd)))
      {
        cpu . interrupt\_flag = sample\_interrupts ();
        cpu . g7\_flag = bG7Pending ();
      }
    // The cpu . wasInhibited accumulates across the even and 
    // odd intruction. If the IC is even, reset it for
    // the next pair.

    if ((PPR . IC % 2) == 0)
      cpu . wasInhibited = false;
\end{verbatim}

If a eligible interrupt is pending, the CPU state is switched to 
INTERRUPT\_cycle.

\begin{verbatim}
    if (cpu . interrupt\_flag)
      {
        setCpuCycle (INTERRUPT\_cycle);
        break;
      }
\end{verbatim} 

Likewise, if a Group 7 faults is pending, cause a fault.

\begin{verbatim}
    if (cpu . g7\_flag)
      {
        cpu . g7\_flag = false;
        doG7Fault ();
      }
\end{verbatim}

There is now code to process the XEC and XED instructions; the idea here is
that the processing of the XEC and XED instructions loads the target 
instructions into the Control Unit IWB and IODD words, and that the 
fetch cycle is a no-op, as the instructions have already been 
fetched.

If not the XEC or XED the case, the instruction is fetched into the
CU IWB word.

\begin{verbatim}
    else
      {
        processorCycle = INSTRUCTION\_FETCH;
        clr\_went\_appending ();
        fetchInstruction (PPR . IC);
      }
\end{verbatim}

Now that the instruction is in the IWB, switch to EXEC state.

\begin{verbatim}
    setCpuCycle (EXEC\_cycle);
    break;
\end{verbatim}

\section{EXEC\_cycle}

The execute cycle starts right off with:

\begin{verbatim}
    t\_stat ret = executeInstruction ();
\end{verbatim}

A return value of greater than 0 indicates that the CPU needs to halt:

\begin{verbatim}
    if (ret > 0)
      {
        reason = ret;
        break;
      }
\end{verbatim}

A return value of CONT\_TRA indicates that a transfer instruction 
was executed, which requires different handling. A transfer 
may be of interest to the handling of
transfer into append or BAR modes, so the `wasXfer' flag is set; and
the code that increments the IC is skipped. The CPU state is set to
EXEC\_cycle, and it all repeats.

\begin{verbatim}
    if (ret == CONT\_TRA)
      {
        cpu . wasXfer = true;
        setCpuCycle (FETCH\_cycle);
        break;   // don't bump PPR.IC, instruction already did it
      }
    cpu . wasXfer = false;
\end{verbatim}


Next, the IC is incremented to point to the next instruction; any
EIS operands that the instruction had are also skipped over.
The CPU state is set to FETCH\_cycle, and it all repeats.

\begin{verbatim}
    PPR.IC ++;
    if (ci->info->ndes > 0)
      PPR.IC += ci->info->ndes;

    cpu . wasXfer = false;
    setCpuCycle (FETCH\_cycle);
    break;
\end{verbatim}


\section{Instruction execution: `executeInstruction'}

Instruction execution is managed by `executeInstruction()'.

When `executeInstruction' starts, the instruction has been loaded
into the CU IWB (Control Unit Instruction Working Buffer).

First, the instruction is decoded, extracting the operation code, address
field, tag field and the A and I bits into the `currentInstruction'
structure; and setting the 'info' member to point to the operation's
entry in the opcode table.

\begin{verbatim}
    decodeInstruction(cu . IWB, ci);
\end{verbatim}

The 'info' member contains a wide variety of details about the instruction,
including:

\begin{description}
\item [PREPARE\_CA] Instruction will need the operand's computed address.
\item [READ\_OPERAND] Instruction will read the operand.
\item [WRITE\_OPERAND] Instruction will write the operand.
\item [NO\_RPT] Not allowed in a repeat instruction.
\item [PRIV\_INS] Privileged instruction.
\item [] The number of EIS operands.
\item [] Allowed tag values.
\end{description}

The flags in the `info' structure are used to check instruction restrictions, 
such as privilege and allowed modifiers; violation causes a fault.

\begin{verbatim}
    if ((ci -> info -> flags & PRIV\_INS) && ! is\_priv\_mode ())
        doFault (FAULT\_IPR, ill\_proc, 
                 "Attempted execution of privileged instruction.");
    // No CI/SC/SCR allowed
    if (ci->info->mods == NO\_CSS)
    {
        if (\_nocss[ci->tag])
            doFault(FAULT\_IPR, ill\_mod, "Illegal CI/SC/SCR modification");
    }
    // No DU/DL/CI/SC/SCR allowed
    else if (ci->info->mods == NO\_DDCSS)
    {
        if (\_noddcss[ci->tag])
            doFault(FAULT\_IPR, ill\_mod, "Illegal DU/DL/CI/SC/SCR modification");
    }
    // No DL/CI/SC/SCR allowed
    else if (ci->info->mods == NO\_DLCSS)
    {
        if (\_nodlcss[ci->tag])
            doFault(FAULT\_IPR, ill\_mod, "Illegal DL/CI/SC/SCR modification");
    }
    // No DU/DL allowed
    else if (ci->info->mods == NO\_DUDL)
    {
        if (\_nodudl[ci->tag])
            doFault(FAULT\_IPR, ill\_mod, "Illegal DU/DL modification");
    }
\end{verbatim}

Next, an assortment of initializations occurs, setting various registers
to the operand address field, and initializing the TPR register.

\begin{verbatim}
    TPR.CA = address;
    iefpFinalAddress = TPR . CA;
    rY = TPR.CA;

    TPR.TRR = PPR.PRR;
    TPR.TSR = PPR.PSR;
\end{verbatim}

If the instruction has EIS operands, they are read into holding variables.

\begin{verbatim}
    if (info -> ndes > 0)
      {
        for(int n = 0; n < info -> ndes; n += 1)
          {
// XXX This is a bit of a hack; In general the code is good about 
// setting up for bit29 or PR operations by setting up TPR, but
// assumes that the 'else' case can be ignored when it should set
// TPR to the canonical values. Here, in the case of a EIS instruction
// restart after page fault, the TPR is in an unknown state. Ultimately,
// this should not be an issue, as this folderol would be in the DU, and
// we would not be re-executing that code, but until then, set the TPR
// to the condition we know it should be in.
            TPR.TRR = PPR.PRR;
            TPR.TSR = PPR.PSR;
            Read (PPR . IC + 1 + n, & ci -> e . op [n], EIS\_OPERAND\_READ, 0);
          }
        // This must not happen on instruction restart
        if (! (cu . IR & I\_MIIF))
          {
            du . CHTALLY = 0;
            du . Z = 1;
          }
      }
\end{verbatim}

If the instruction expects the address field to be converted to the
Computed Address, do that.

\begin{verbatim}
        if (ci->info->flags & PREPARE\_CA)
          {
            doComputedAddressFormation ();
            iefpFinalAddress = TPR . CA;
          }
\end{verbatim}

Otherwise if the instruction wants the operand value, do that. `readOperands' will handle single, double, eight and sixteen word operands. The value is held in `CY', `Ypair', `Yblock8' or `Yblock16' holding registers, as appropriate. Read operands handles all aspects of indirection and address appending.

\begin{verbatim}
        else if (READOP (ci))
          {
            doComputedAddressFormation ();
            iefpFinalAddress = TPR . CA;
            readOperands ();
          }
\end{verbatim}

Now that the operands have sorted, the instruction is executed.

\begin{verbatim}
    t\_stat ret = doInstruction ();
\end{verbatim}

If an transfer instruction has the A bit set, and accesses the Append Unit
during the Computed Address formation, the processor is switch to Append
mode.

\begin{verbatim}
    if (info->ndes == 0 && a && (info->flags & TRANSFER\_INS))
      {
        if (get\_addr\_mode () == BAR\_mode)
          set\_addr\_mode(APPEND\_BAR\_mode);
        else
          set\_addr\_mode(APPEND\_mode);
      }
\end{verbatim}

Finally, it the instruction writes the operand, do that.

\begin{verbatim}
    if (WRITEOP (ci))
      {
        if (! READOP (ci))
          {
            doComputedAddressFormation ();
            iefpFinalAddress = TPR . CA;
          }
        writeOperands ();
      }
\end{verbatim}

\section{Instruction execution: `doInstruction'}

First, initialize the EIS state registers.

\begin{verbatim}
    if (i->info->ndes > 0)
    {
        i->e.ins = i;
        i->e.addr[0].e = &i->e;
        i->e.addr[1].e = &i->e;
        i->e.addr[2].e = &i->e;

        i->e.addr[0].mat = OperandRead;   // no ARs involved yet
        i->e.addr[1].mat = OperandRead;   // no ARs involved yet
        i->e.addr[2].mat = OperandRead;   // no ARs involved yet
    }
\end{verbatim}

And switch based on the opcode extension bit; `Basic' and `EIS' are
misleading here; the opcode extension bit has a minimal correlation to
the EIS instruction opcode layout, but by separating the two cases,
the code becomes a bit more readable, Both routines have the same
preamble:

\begin{verbatim}
    DCDstruct * i = & currentInstruction;
    uint opcode  = i->opcode;  // get opcode

    switch (opcode)
    {
       ....
    }
\end{verbatim}

We will look at a few of the instructions so as to understand the
general function of 'doInstruction.'

\subsection{LCA Load Complement A}

`readOperands()' has already retrieved the operand value, and left it in CY.
The utility routine `compl36' complements the value and sets the IR flags,
and the assignment operation places the complemented value in the A register.

\begin{verbatim}
    case 0335:  // lca
      rA = compl36 (CY, & cu . IR);
      break;
\end{verbatim}

\subsection{LREG Load Registers}

LREG loads the A, Q, E, and index registers from a Y-block8. Again, 
`readOperands()' has loaded the values into Yblock8.

\begin{verbatim}
    case 0073:   ///< lreg
      rX[0] = GETHI(Yblock8[0]);
      rX[1] = GETLO(Yblock8[0]);
      rX[2] = GETHI(Yblock8[1]);
      rX[3] = GETLO(Yblock8[1]);
      rX[4] = GETHI(Yblock8[2]);
      rX[5] = GETLO(Yblock8[2]);
      rX[6] = GETHI(Yblock8[3]);
      rX[7] = GETLO(Yblock8[3]);
      rA = Yblock8[4];
      rQ = Yblock8[5];
      rE = (GETHI(Yblock8[6]) >> 10) & 0377;   // need checking
      break;
\end{verbatim}

\subsection{STA Store A}

Since `writeOperands()' will do the actual writing, all STA needs to
do is copy A to CY.

\begin{verbatim}
    case 0755:  // sta
      CY = rA;
      break;
\end{verbatim}

\subsection{STXn Store Index Register n}

For many opcodes, the low bits of the opcode contains indexing 
information.

\begin{verbatim}
    case 0740:  ///< stx0
    case 0741:  ///< stx1
    case 0742:  ///< stx2
    case 0743:  ///< stx3
    case 0744:  ///< stx4
    case 0745:  ///< stx5
    case 0746:  ///< stx6
    case 0747:  ///< stx7
      {
        uint32 n = opcode & 07;  // get n
        SETHI(CY, rX[n]);
      }
    break;
\end{verbatim}

\subsection {TRA Transfer}

For the TRA instruction, the computed address is placed in the PPR, and the
function return value is used to signal that a transfer is to occur.

\begin{verbatim}
    case 0710:  ///< tra
      PPR.IC = TPR.CA;
      PPR.PSR = TPR.TSR;
      return CONT\_TRA;
\end{verbatim}

\section{RPT/RPD}

TODO

\section{XEC/XED}

TODO

\section{EIS}

TODO

\section{Computed Address Formation}

TODO

\section{Append Unit}

TODO

\section{Interrupts}

Interrupts are handled at the start of the CPU fetch cycle:

\begin{verbatim}
    if (cpu . interrupt\_flag)
      {
        setCpuCycle (INTERRUPT\_cycle);
        break;
      }
\end{verbatim}

The DPS8M interrupt handling logic is to save the system state, place the processor in "temporary absolute 
mode", fetch a pair of instructions from the indicated location in memory and execute them. If one
of the pair of instructions is a transfer instruction, the processor is set to absolute mode. If
neither of the instructions transfers, the Control Unit is restored from the saved state and the
processor is switched back to the fetch cycle.

The INTERRUPT\_cycle handler saves the Control Unit state so the processor can return to the
state that was extant at the time the interrupt was handled. The interrupt number being serviced
is stored in the Control Unit, where the guest operating system can inspect it.

\begin{verbatim}
  // In the INTERRUPT CYCLE, the processor safe-stores
  // the Control Unit Data (see Section 3) into 
  // program-invisible holding registers in preparation 
  // for a Store Control Unit (scu) instruction, enters 
  // temporary absolute mode, and forces the current 
  // ring of execution C(PPR.PRR) to
  // 0. It then issues an XEC system controller command 
  // to the system controller on the highest priority 
  // port for which there is a bit set in the interrupt 
  // present register.  

  uint intr\_pair\_addr = get\_highest\_intr ();
  cu . FI\_ADDR = intr\_pair\_addr / 2;
  cu\_safe\_store ();
\end{verbatim}

Next, the processor is placed in  "temporary absolute mode":

\begin{verbatim}
    // Temporary absolute mode
    set\_TEMPORARY\_ABSOLUTE\_mode ();

    // Set to ring 0
    PPR . PRR = 0;
    TPR . TRR = 0;
\end{verbatim}

The interrupt pair is fetched and scheduled for execution:

\begin{verbatim}
    // get interrupt pair
    core\_read2 (intr\_pair\_addr, instr\_buf, instr\_buf + 1, \_\_func\_\_);

    cpu . interrupt\_flag = false;
    setCpuCycle (INTERRUPT\_EXEC\_cycle);
    break;
\end{verbatim}

The INTERRUPT\_EXEC\_cycle handler recovers an instruction from holding and executes it.

\begin{verbatim}
    case INTERRUPT\_EXEC\_cycle:
    case INTERRUPT\_EXEC2\_cycle:
      {
        if (cpu . cycle == INTERRUPT\_EXEC\_cycle)
          cu . IWB = instr\_buf [0];
        else
          cu . IWB = instr\_buf [1];

        t\_stat ret = executeInstruction ();
\end{verbatim}

If the instruction was a transfer instruction, set the processor to absolute mode and start
normal fetch/execute processing.

\begin{verbatim}
    if (ret == CONT\_TRA)
      {
        cpu . wasXfer = true;
        setCpuCycle (FETCH\_cycle);
        set\_addr\_mode (ABSOLUTE\_mode);
        break;
      }
\end{verbatim}

Otherwise, if the instruction just executed was the first of the pair, schedule the execution of
the second.

\begin{verbatim}
    if (cpu . cycle == INTERRUPT\_EXEC\_cycle)
      {
        setCpuCycle (INTERRUPT\_EXEC2\_cycle);
        break;
      }
\end{verbatim}

The only possibility now is that both instructions have been executed and neither transferred, so restore
the saved state and resume processing.

\begin{verbatim}
    clear\_TEMPORARY\_ABSOLUTE\_mode ();
    cu\_safe\_restore ();
    cpu . wasXfer = false;
    setCpuCycle (FETCH\_cycle);
    break;
\end{verbatim}
 
\section{Faults}

Faults fall (mostly) into two categories, ones that are generated by instruction execution (page faults,
overflow, etc.), and those that are independent of the instruction (timer run-out, lockup fault, etc.).


The Group 7 faults are handled similarly to interrupts; at the start of the CPU fetch cycle the
Group 7 fault pending flag is queried.

\begin{verbatim}
    if (cpu . g7\_flag)
      {
        cpu . g7\_flag = false;
        doG7Fault ();
      }
\end{verbatim}

`doG7Fault' is invoked rather then switching 
cycles directly.

\begin{verbatim}
void doG7Fault (void)
  {
     if (g7Faults & (1u << FAULT\_TRO))
       {
         g7Faults &= ~(1u << FAULT_TRO);
         doFault (FAULT\_TRO, 0, "Timer runout");
       }

     if (g7Faults & (1u << FAULT\_CON))
       {
         g7Faults &= ~(1u << FAULT\_CON);
         cu . CNCHN = g7SubFaults [FAULT\_CON] & MASK3;
         doFault (FAULT\_CON, g7SubFaults [FAULT\_CON], "Connect");
       }

     // Strictly speaking EXF isn't a G7 fault, put if we treat is as one,
     // we are allowing the current instruction to complete, simplifying
     // implementation
     if (g7Faults & (1u << FAULT\_EXF))
       {
         g7Faults &= ~(1u << FAULT\_EXF);
         doFault (FAULT\_EXF, 0, "Execute fault");
       }
     doFault (FAULT\_TRB, (\_fault\_subtype) g7Faults, "Dazed and confused in doG7Fault");
  }
\end{verbatim}

Both the interrupt and fault paths involve saving the Control Unit, but the fault case was abstracted into
`doFault', due to the large number of code paths leading there; while there is only the single entry into
the INTERRUPT\_cycle, and the Control Unit save is done in the INTERRUPT\_cycle state handler.

TODO

\subsection{doFault()}

TODO

\subsection{RCU}

TODO

\part{"As-built"}

\section{Variations between the hardware and the emulator}

DPS8 memory is managed by the CPU code, not the SCU code.

The history registers are not implemented.

The RPL instruction is not implemented.

The MPCs have been abstracted into the device code.

\end{document}
